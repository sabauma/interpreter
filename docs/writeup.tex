\documentclass[8pt]{article}

\usepackage{amsmath}
\usepackage{amsfonts}
\usepackage{hyperref}

\author{Spenser Bauman}
\title{CSCI-B629 Writeup}
\date{\today}

\begin{document}
\maketitle

\section{Truffle Interpreter}

This portion of the project consisted of making an interpreter for a Scheme like
language using the Truffle DSL.
Truffle consists of several components which handle the details of tree rewriting
in an AST interpreter.
This document will attempt to outline the major components and how they are used
to implement this interpreter.
The basic implementation strategy was based off of a Truffle
tutorial\footnote{\url{https://cesquivias.github.io/blog/2014/10/13/writing-a-language-in-truffle-part-1-a-simple-slow-interpreter/}}.

My initial strategy was to write an AST interpreter and then transition it over
to using Truffle, hoping that the modifications could be made gradually -- similar
to an RPython interpreter.
The transitioning took more work that was expected, as Truffle requires all AST
nodes to be capable of performing tree rewriting.

\subsection{Truffle Annotations}

The Truffle DSL consists largely of a set of Java annotations.
These annotations then generate many of the class members for managing tree
rewriting and specialized nodes from a class specification.
The user specifies which components of an AST node correspond to child nodes,
and largely allows the user to ignore the low level details of tree rewriting.

The biggest challenge with using Truffle is learning to phrase optimizations
as tree rewriting.
For instance, this interpreter uses tree rewriting to speed up variable lookup
by rewriting variable nodes which traverse the environment to a version with
a cached environment frame.

The annotations also generate the class definitions for the specialized versions
of a node.
Manually implementing each class would require much more work, resulting in several
times more class definitions in the implementation.

\subsection{Environment Management}

To facilitate type specialization of environment accesses (and communicate this
information to Graal), Truffle defines several \texttt{Frame} types.
During parsing, the user must generate a \texttt{FrameDescriptor} which
specifies the shape of the environment frames.
Lexical scoping is obtained by linking frames into a list like structure and
storing them on closures.

This approach is somewhat more restrictive than RPython, requiring up front
processing and exclusive use of the Truffle API.
That said, Truffle frames contain slots for unboxed values, so the end result is
unboxing equivalent to how Pycket unboxes values in environment frames -- without
manual intervention.

\subsection{Procedure Calls}

The most difficult functionality to implement was procedure calls.
Truffle has a notion of a ``call target'', which is created when building
a closure, wrapping up the code for the function.
Similarly, application nodes include a call node which is responsible for invoking
the call target after.

Truffle includes several types of call nodes, depending on how procedure calls work
in the language.
Some call nodes may be better optimized by Truffle and Graal, at the cost of
more implementation work.
The interpreter currently uses the \texttt{IndirectCallNode}, as it is the easiest
to use.

One difficulty with implementing Scheme on top of Java as an AST interpreter is
the implementation of tail calls.
Application nodes in tail position will allocate a Java stack frame, so a reliable
Scheme implementation needs a work around.
When an application in tail position occurs, a \texttt{TailCallException} is
thrown, which wraps the call target and arguments.
Function applications not in tail position consist of loops which catch tail call
exceptions and execute the wrapped function.

This is an inelegant way to implement tail calls, but is necessitated by the
structure of Truffle interpreters.
This is definitely a case where less constrained nature of RPython interpreters
wins out.

\subsection{Conclusion}

While the performance of the interpreter has gotten much better by using Truffle,
it is still over an order of magnitude slower than Pycket on simple benchmarks.
Truffle adds in the logic needed to use the Graal JIT, which provides huge
performance improvements of the standard JVM, though I have not yet managed to
perform serious benchmarking, and I have only performed a few optimizations.

At this point, I believe it is possible to produce an interpreter competitive
with Pycket using Truffle, though more work may be required to achieve similar
functionality -- I have no idea how to go about implementing features like
\texttt{call/cc} though escape continuations may not be too difficult.
(R)Python's metaprogramming facilities are more usable than Java's, which greatly
simplifies a lot of the implementation work of Pycket.
On the other hand, the fact that Truffle (including the automatic code generation)
works with Eclipse is a big plus in favor of the Truffle approach.

\section{Partial Evaluator}

This part of the project made use of the \texttt{pgg} partial evaluator by Mike
Sperber et al.
\texttt{pgg} is a partial evaluator for the Scheme 48 system.
The general goal of this project was to explore how well the partial evaluator
could remove the interpretation overhead of a simple definitional interpreter when
specializing on a program.
The results were mixed, as the partial evaluator seems to be rather sensitive to
the code structure and some of the hints provided by the partial evaluator.

The partial evaluator did not perform as well on the interpreters I devised as it
did on their example interpreters.
The example interpreter that came with \texttt{pgg} typically had all of the
environment management elided in the resulting code, which I have not been able
to manage.

That said, \texttt{pgg} managed to eliminate most of the overhead from the
interpreter on simple programs.
All of the overhead of dispatching on the AST structure was elided from the
final result.

My initial attempts were straightforward definitional interpreters using a list
of pairs for the environment representation.
Depending on the program, often resulted in environment management and AST
dispatch in the residual code -- typically anything with a \texttt{lambda} would
produce this behaviour.
Two things improved the initial versions of the code: primitive annotations and
capturing variables rather than passing them around.
The \texttt{pgg} system has some annotations that tell it how to treat primitive
operations like \texttt{apply} and \texttt{cons}.
Annotating \texttt{cons} as pure, for instance, improved the residual code for
environment management.
The proper annotations for \texttt{apply} allow \texttt{pgg} to convert uses of
the \texttt{apply} function to normal function calls when the arity is known
statically.

The \texttt{pgg} documentation suggested that using custom data types, rather than
\texttt{cons} cells, might work better for an environment representation, but
that had little effect on the residual code.
One annotation that improved the resulting code (when it did not result in
non-termination of \texttt{pgg}) was to define the interpretation function without
memorization, which should be safe when the functions termination depends only on
static data.
This annotation typically resulted in the partial evaluator exploring error paths
in environment lookup, though those branches are dead code.

In an attempt to remove the overhead of environment traversal, I produced
a manually staged version of the interpreter.
To do so, the environment is split into static and dynamic portions
The interpretation function then takes the input expression and static environment
and produces a function from the dynamic environment to the result.
Staging the interpreter allowed the partial evaluator to remove symbol comparisons
from the residual code, though not the traversals of the dynamic environment.

Overall, this seems like an unreliable method for implementing a high performance
compiler from an interpreter.
It is difficult to assess ahead of time how the partial evaluator will behave and
little diagnostic information to address the problem
It is difficult to assess ahead of time how the partial evaluator will behave and
little diagnostic information to address the problem.

\end{document}

